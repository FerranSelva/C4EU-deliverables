The idea of an open access network to which anyone can connect appers in \cite{battiti2003ggo}.
The underlying concept is that the value of a network grows as its size grows.
Therefore, the organic growth of open access networks benefits all the participants.

There are examples in which alternative models for broadband network deployment have helped in the digital development of entire regions.
In \cite{oliver2010wca}, the case of \emph{guifi.net} in Catalonia is presented.

The Swedish experience is described in \cite{forzati2010oan}, that covers the evolution of their open access models over ten years.
In the Swedish business, model three different layers are differentiated.
The lower layer includes passive elements such as fiber closets and its owned by the ``network operator''.
On top of that passive layer, there is the layer of active elements (routers, switches) that are owned and operated by the ``communications operator''.
Finally, the upper layer is the ``service provider'' which offers content and Internet access.
The model makes it possible that different network operators share the same infrastructure and compete with each other.
Similarly, several service providers can offer their services on top of a communications operator.

A similar idea is proposed in \cite{cohill2010ttw}.
The argument is that the solution is not a completely private or a completely public model.
The paper takes the analogy of roads to explain that the infrastructure should be owned by the community and shared to offer both public and private services.
