\IEEEPARstart{S}{uper} WiFi is an emerging attempt to make use of the available spectrum in the UHF TV band (between $471.25$ and $863.25$~MHz) for wireless data transmissions. This spectrum availability is the result from the digital switchover of analog TV channels, allowing more TV data to be transmitted over the same $8$~MHz-width UHF TV channel.

Some of the benefits of attempting transmission over these frequencies are longer ranges and better building penetration than in the unlicensed $2.4$ and $5$~GHz bands used by the IEEE~$802.11$ set of protocols. These benefits are followed by strict regulatory and technological challenges regarding the administration of TV White Spaces (TVWS) and cognitive capabilities for incumbent avoidance.

This report assesses some of Super WiFi's technical challenges, specially those related to incumbent detection and avoidance~\cite{shellhammer2009technical}. As a result, an energy detector on the UHF TV band is implemented using the Universal Software Radio Peripheral (USRP) Ettus USRP-E110~\cite{ettusUSRPE110} as a radio front end for embedded applications (see Figure~\ref{fig:usrp_combined}) and the open source Software Defined Radio (SDR) project GNURadio~\cite{GNURadio}. Furthermore, an instant image of the TV UHF band is derived from the detection (see Figure~\ref{fig:tvChannels}).

Section~\ref{sec:related_work} overviews previous work in this area, followed by Section~\ref{sec:implementation} which details the implementation of a spectrum sensor using the USRP-E110. Results are summarized in Section~\ref{sec:results} while conclusions and future directions are provided in Section~\ref{sec:conclusions}.