The proposed identification of TV White Spaces with USRP-E110 enables the execution of a spectrum sensing algorithm via SSH, allowing the USRP to be located at convenient locations.

It is possible to build a Radio Environment Maps (REM)~\cite{zhao2007applying} from samples gathered by geographically distributed USRPs controlled from a centralized location, increasing the efficiency and boosting the implementation of cognitive networks.

In order optimize the spectrum sensing algorithm, better signal processing techniques are expected to be implemented in the near future~\cite{shellhammer2009technical}. All of this in the attempt to differentiate noise from TV broadcast signals.

Currently, our research is oriented towards the effective communication of two USRP-E110 using TVWS at distances over three meters apart~\cite{sanabriaUSRPTX}. Also, we are working at combining the cognitive and transmission tasks inside a unified code running in the USRP-E110.